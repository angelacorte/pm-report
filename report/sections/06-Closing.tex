\section{Closing}


\subsection{Approvazione per la chiusura del progetto}
Per la fase di chiusura del progetto, è necessario che il committente approvi il lavoro svolto.\\

È stato effettuato un incontro con il committente per verificare che il prodotto soddisfi i requisiti richiesti e che il progetto sia stato svolto secondo le tempistiche e i costi preventivati.\\

Il lavoro completato è stato inviato in anticipo rispetto al meeting, in modo tale che il committente potesse esaminarlo con calma e prepararsi per l'incontro o eventualmente apportare qualche modifica in vista del meeting finale.\\


\subsection{Audit post-implementazione}
In seguito all'accettazione del prodotto da parte degli stakeholders, è stato effettuato un meeting per delineare tutte le informazioni dei punti principali del progetto all'interno di un documento.\\

Questo documento è stato redatto in modo tale da poter essere utilizzato come riferimento per eventuali progetti futuri, in modo tale da poter velocizzare la fase di analisi e progettazione.
Inoltre, sono state inserite all'interno del documento le scelte implementative effettuate con le motivazioni e i problemi riscontrati durante la fase di sviluppo.\\

Il meeting è utile anche per il team, in quanto si discute dei punti di forza e di debolezza del loro lavoro e si controlla che effettivamente i vari requisiti del progetto siano stati soddisfatti.\\

\subsection{Installazione della soluzione}
Il progetto per natura non ha prodotto un software eseguibile, bensì sono state create varie librerie che possono essere utilizzate all'interno di altri progetti a seconda delle esigenze dei programmatori.\\

Per utilizzare tali librerie è necessario aggiungerle nelle dipendenze del proprio progetto e importarle all'interno del codice.\\


\subsection{Chiusura del progetto}
Il progetto si considera ufficialmente terminato nel momento in cui gli stakeholders controllano e firmano il documento finale. In più è necessario
che i Project Manager ottengano la documentazione utile per eventuali progetti futuri o manutenzioni del prodotto.\\

In allegato al report finale vengono aggiunti tutti i documenti prodotti durante lo sviluppo del progetto, in modo tale da
attestare la qualità del lavoro svolto e le tecniche utilizzate per la realizzazione del prodotto.\\

Di questo report finale infine ne viene consegnata una copia anche agli stakeholders.