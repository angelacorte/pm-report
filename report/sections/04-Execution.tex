\section{Launching and Execution}

La fase di lancio ed esecuzione in un progetto è cruciale per trasformare la pianificazione in azione. 
Durante questa fase, si seleziona il personale da coinvolgere nel team di progetto, si stabiliscono le regole operative del team 
e si aiuta il team a lavorare insieme in modo efficiente per raggiungere gli obiettivi stabiliti.

\subsection{Team di Sviluppo}

Successivamente alla fase di pianificazione, è stato definito il team di sviluppo, composto da:

\begin{itemize}
    \item 2 project manager;
    \item 2 senior developer;
    \item 3 junior developer;
    \item 1 sistemista;
    \item Il client team, composto da esperti del dominio.
\end{itemize}

Il team di sviluppo è già consolidato, con esperienze di lavoro passate insieme, eliminando la necessità di un periodo di rodaggio. Per fornire un supporto continuo e fondamentale nella comprensione dei problemi in fase di sviluppo, il team ha coinvolto esperti del dominio dal lato dei clienti.

Durante la fase di pianificazione e definizione degli obiettivi, i Project Manager e i senior developer hanno svolto un ruolo attivo, mentre i junior developer sono stati coinvolti principalmente nelle fasi di sviluppo.

\subsection{Kick-off Meeting}

Il kick-off meeting rappresenta l'inizio effettivo della fase di esecuzione del progetto ed è stato progettato con un'agenda ben definita per informare tutti i membri del team di sviluppo sulle attività da svolgere.

\subsubsection{Agenda}

L'agenda del kick-off meeting comprendeva i seguenti punti:

\begin{itemize}
    \item Presentazione del team di sviluppo;
    \item Presentazione del team di clienti (client team);
    \item Presentazione del progetto e degli obiettivi;
    \item Presentazione della pianificazione;
    \item Presentazione delle regole operative per il team;
    \item Presentazione del piano per la qualità;
    \item Creazione di una schedula di lavoro basata sulle disponibilità dei membri del team;
    \item Sessione di domande e risposte.
\end{itemize}

\subsubsection{Regole operative per il team}

Le regole operative sono state stabilite per garantire una gestione efficace delle attività del team durante l'esecuzione del progetto.

\paragraph{Riunioni}

\begin{itemize}
    \item Deve essere programmata almeno una riunione settimanale, con inclusione di una Project Review Meeting all'interno di ogni Sprint;
    \item Ogni riunione settimanale deve includere la discussione dei risultati ottenuti nella Sprint precedente, dei problemi riscontrati e della presentazione dei task da completare per raggiungere le milestone;
    \item È possibile organizzare riunioni straordinarie, denominate Problem Resolution Meetings, in caso di emergenze o necessità;
    \item È previsto un Daily Status Meeting di 15 minuti ogni giorno per pianificare la giornata lavorativa.
\end{itemize}

\paragraph{Modalità di comunicazione}

\begin{itemize}
    \item Le comunicazioni all'interno del team devono avvenire attraverso la piattaforma Slack, specificamente nel canale dedicato al progetto;
    \item Ogni attività completata deve essere aggiornata nel Project Board su GitHub;
    \item Tutta la documentazione relativa al progetto deve essere creata e gestita tramite GitHub;
    \item È obbligatorio confermare la ricezione di comunicazioni importanti.
\end{itemize}

\paragraph{Strategie di version control}

\begin{itemize}
    \item È stata creata un'organizzazione su GitHub, contenente repository dedicati ai documenti e al codice;
    \item Ciascun sotto-progetto ha il proprio repository, con issue e progetti dedicati alla gestione del lavoro;
    \item Ogni repository dispone di un branch "main" su cui non è consentito lavorare e di un branch "nomeFeature" su cui apportare modifiche;
    \item Ogni membro del team deve creare un branch dedicato per ciascuna issue assegnata e aprire una pull request per il merge;
    \item Ogni pull request deve essere approvata da uno sviluppatore senior.
\end{itemize}

\subsubsection{Piano per la qualità}

Per garantire l'eccellenza tecnica, sono state stabilite le seguenti pratiche:

\begin{itemize}
    \item Utilizzo di uno strumento di build per la gestione delle dipendenze dalle librerie esterne;
    \item Utilizzo di una lingua inglese esclusivamente per la descrizione dei commit;
    \item Standardizzazione dei commit secondo le linee guida dei Conventional Commits;
    \item Implementazione di un flusso di lavoro per il testing automatico della base di codice;
    \item Inclusione della documentazione all'interno del repository di ciascun sotto-progetto;
    \item Commenti nel codice conformi alle convenzioni del linguaggio utilizzato;
    \item Adozione di pattern di programmazione consolidati per garantire riutilizzabilità, scalabilità ed estensibilità del codice;
    \item Creazione di un flusso di lavoro per la valutazione della copertura del codice.
\end{itemize}

\subsection{Sprint}

È stato redatto un documento dedicato per la pianificazione delle sprint (documento "Sprints"), in cui sono stati definiti i task da svolgere, le risorse necessarie e le tempistiche per garantire il raggiungimento degli obiettivi.
