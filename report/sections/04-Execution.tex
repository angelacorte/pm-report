\section{Launching and Execution}
%si seleziona il personale da coinvolgere nel team di porgetto, si stabiliscono le regole operative del team e si aiuta il team a lavorare insieme

\subsection{Team di Sviluppo}
Successivamente alla fase di planning, è stato definito il team di sviluppo, composto da:
\begin{itemize}
    \item 2 project manager;
    \item 2 senior developer;
    \item 3 junior developer;
    \item 1 Sistemista;
    \item client team (esperti del dominio).
\end{itemize}

Il team è oramai consolidato, e si è già lavorato insieme in passato, quindi non è stato necessario un periodo di rodaggio.
Lato clienti invece sono stati coinvolti esperti del dominio, per fornire un supporto fondamentale e continuo per la comprensione
degli eventuali problemi in fase di sviluppo.

Del personale citato, hanno preso parte alle fasi di planning e scoping i Project Manager e i senior developer, mentre i junior developer
sono stati coinvolti nelle fasi di sviluppo.

\subsection{Kick-off Meeting}
Da questa riunione si rappresenta l'effettivo inizio della fase di esecuzione del progetto.

\subsubsection{Agenda}
L'agenda del kick-off meeting è stata definita in maniera tale da informare tutti i membri del team di sviluppo sulle attività
da svolgere.

\begin{itemize}
    \item Resoconto:
          \begin{itemize}
              \item Presentazione del team di sviluppo;
              \item Presentazione del team di clienti;
              \item Presentazione del progetto;
              \item Presentazione del planning;
              \item Presentazione delle regole operative per il team;
              \item Presentazione del piano per la qualità;
              \item Creazione di una schedula di lavoro a seconda delle disponibilità dei membri del team;
              \item Domande e risposte.
          \end{itemize}
\end{itemize}

\subsubsection{Regole operative per il team}
\paragraph{Riunioni}
\begin{itemize}
    \item Almeno una riunione settimanale, con redazione di Project Review Meeting (compreso nelle Sprint);
    \item Discussione di risultati ottenuti dalla sprint precedente ed eventuali problemi riscontrati;
    \item Presentazione di task da terminare al fine di raggiungere le milestone;
    \item Possibilità di organizzare riunioni straordinarie in caso di problemi o necessità (Problem Resolution Meetings);
    \item Daily status meeting di durata di 15 minuti per pianificare la giornata.
\end{itemize}

\paragraph{Modalità di comunicazione}
\begin{itemize}
    \item Per comunicare sarà necessario farlo attraverso la piattaforma Slack nel canale apposito;
    \item Al completamento di ogni attività, sarà necessario aggiornare il Project Board di GitHub;
    \item I documenti verranno redatti sempre attraverso GitHub;
    \item I membri sono tenuti a fornire conferme di ricezioni per le comunicazioni importanti.
\end{itemize}

\paragraph{Strategie di version control}

\begin{itemize}
    \item È stata creata un'organizzazione su GitHub, in cui sono stati creati i repository per i documenti e per il codice;
    \item Ogni sotto-progetto ha il proprio repository, in cui sono presenti le issue e i progetti per la gestione del lavoro;
    \item In ogni repository sarà presente un branch \textit{main} su cui è vietato lavorare, e un branch \textit{nomeFeature} su cui verranno effettuate le modifiche;
    \item Ogni membro del team è tenuto a creare un branch per ogni issue assegnata, e a fare una pull request per il merge;
    \item Ogni pull request deve essere approvata da uno sviluppatore senior.
\end{itemize}

\subsubsection{Piano per la qualità}

Devono essere adottate le seguenti pratiche per raggiungere l'eccellenza tecnica:

\begin{itemize}
    \item Utilizzo di uno strumento di build per la gestione delle dipendenze dalle librerie esterne;
    \item Descrizione dei commit esclusivamente in lingua inglese;
    \item Standardizzazione nella formattazione dei commit seguendo lo standard dei Conventional Commits;
    \item Implementazione di un flusso di lavoro per il testing automatico della base di codice;
    \item Inclusione della documentazione all'interno del repository del sotto-progetto;
    \item Inserimento di commenti nel codice in conformità con le convenzioni del linguaggio utilizzato;
    \item Adozione di pattern di programmazione consolidati per garantire la riutilizzabilità, la scalabilità e l'estensibilità del codice;
    \item Creazione di un flusso di lavoro per la valutazione della copertura del codice.
\end{itemize}



\subsection{Sprint}
È stato redatto un documento apposito per la pianificazione delle sprint (documento \textbf{"Sprints"}), in cui sono stati definiti i task da svolgere, le risorse necessarie e le tempistiche.