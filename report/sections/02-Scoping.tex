\section{Scoping and Initiation}

% sviluppa ed ottiene l'approvazione di una dichiarazione generale riguardante obiettivi e business value del progetto.
% in questa fase si sceglie se il "cosa" da realizzare va bene a tutti.

Dalla volontà di intraprendere una ricerca all'interno dell'università, nasce l'idea di espandere un progetto sviluppato dai docenti.
Si tratta dell'espansione del progetto ScaFi, una libreria e framework che offre funzionalità inerenti all'aggregate computing, ovvero un
approccio alla programmazione che tratta un gruppo di dispositivi distribuiti come un'unica entità, semplificando la progettazione e l'implementazione
di sistemi complessi distribuiti.

Si decide di utilizzare un approccio di tipo \textbf{Domain Driven Development} (DDD), in quanto si è ritenuto potesse essere il modo migliore
per comprendere a fondo il dominio, tenendo vari incontri con esperti del settore e stakeholders.

\subsection{Riunione preliminare}
\begin{itemize}
    \item Partecipanti:
          \begin{itemize}
              \item 1 business manager
              \item 1 project manager
              \item 2 programmatori senior
              \item 1 tecnografo
              \item 2 esperti del dominio
          \end{itemize}
    \item Resoconto:
          \begin{itemize}
              \item Introdotta l'idea del progetto da parte degli stakeholders verso il team di sviluppo
              \item Il team ha svolto varie domande a stakeholders ed esperti del dominio per inquadrare meglio il concetto
              \item Al termine della riunione è stato convenuto di programmare altri incontri con gli esperti data la complessità del dominio
          \end{itemize}
    \item Output:
          \begin{itemize}
              \item Bozza di Ubiquitous Language
              \item Bozza dei casi d'uso
              \item Registrazione della riunione post consenso da parte degli esperti del dominio
          \end{itemize}
\end{itemize}

\subsubsection{Ubiquitous Language (UL) e casi d'uso}
Il termine "Ubiquitous Language" (UL) rappresenta il linguaggio condiviso e onnipresente all'interno di un team
di progetto. Contiene la terminologia specifica per il business, oltre a concetti e termini emersi durante la
modellazione dei casi d'uso.

L'UL non è semplicemente un elenco di parole, ma una guida per l'uso preciso dei
termini che deve essere applicata sia nella comunicazione tra i membri del team sia nell'implementazione del
codice del modello. In questo modo, si assicura che tutti condividano una comprensione comune e precisa del
dominio e dei concetti coinvolti nel progetto.

L'UL redatto si può consultare nel documento \textbf{"Domain Driven Development"}.

\subsubsection{Conditions of Satisfaction (COS)}
Dalla riunione preliminare sono state estratte le \textbf{"Conditions of Satisfaction"}, consultabili nell'omonimo documento.

\subsection{Primo project scoping meeting}
Lo scopo del primo meeting è stato quello di definire le principali problematiche e i principali punti di forza all'interno del dominio.

Siccome il team partiva da concetti già sviluppati, è stato fondamentale approfondirne la conoscenza con gli esperti del dominio e
individuare zone in cui c'è stata carenza tecnica o meno, con l'obiettivo di migliorare queste zone.

Per questo motivo inizialmente non è stata stilata un'agenda da rispettare, bensì gli esperti del dominio hanno fornito al team articoli e documentazione
che avrebbero dovuto consultare prima dell'incontro successivo, in maniera tale da poter effettuare domande più mirate per lo sviluppo vero e proprio.

In questo meeting è stata ritenuta necessaria la presenza del team al completo e di più stakeholders ed esperti possibili, rispettivamente 2 e 3.

Dalle analisi ottenute dopo la riunione preliminare, il team ha presentato qualche possibile soluzione al problema, confrontate poi con gli esperti.
Successivamente sono state discusse ed aggiornate le \textit{CoS}.

Infine è stato indetto un nuovo meeting per discutere dei requisiti.

\subsubsection{Project Overview Statement (POS) e Analisi dei rischi}
Durante il primo meeting sono inoltre stati stilati i documenti \textbf{"Project Overview Statement} e \textbf{Risk Analysis}.

\subsection{Secondo Project Scoping Meeting}
\subsubsection{SWOT Analysis}
\subsubsection{Modello di business}
\subsubsection{Requirement Breakdown Structure (RBS)}

\subsection{Terzo Project Scoping Meeting}
\subsubsection{Modello PMLC}
\subsubsection{Workflow}

\subsection{Proof of concept}
