\section{Scoping and Initiation}

L'inizio di un progetto rappresenta una fase critica che richiede attenzione e pianificazione dettagliata.
Nella situazione attuale, è stato preso in considerazione l'espansione del progetto ScaFi, un framework
dedicato all'aggregate computing. Questa fase, chiamata "Scoping and Initiation," è essenziale per definire
chiaramente gli obiettivi e il valore commerciale del progetto.\\

La decisione di espandere il progetto ScaFi è stata influenzata da un'iniziativa all'interno dell'ambiente
universitario. Il team di project management ha collaborato con docenti e altri esperti per valutare questa
opportunità. Il progetto ScaFi ha lo scopo di semplificare la progettazione e l'implementazione di sistemi
complessi distribuiti, trattando un gruppo di dispositivi distribuiti come un'unica entità.
Questo approccio offre un grande valore nel contesto dell'aggregate computing.\\

Per garantire una comprensione approfondita del dominio, è stato scelto di adottare il
\textbf{Domain Driven Development} (DDD). Questo approccio richiede una stretta collaborazione con esperti
del settore e stakeholders. Questa fase iniziale è stata fondamentale per assicurarsi che tutti i membri
del team e gli stakeholder abbiano una visione chiara dei requisiti e delle aspettative del progetto.\\

\subsection{Riunione preliminare}
Nella fase iniziale, è stata tenuta una riunione preliminare coinvolgendo diversi attori chiave, tra cui un
business manager, un project manager, programmatori senior, un tecnografo, uno stakeholder e un esperto del
dominio. In questa riunione, è stata presentata l'idea del progetto agli stakeholders e sono state poste
domande chiave per definire meglio il contesto. È emerso che il dominio è complesso, e quindi è stato
concordato di programmare ulteriori incontri con gli esperti per acquisire ulteriori dettagli.\\

L'output di questa fase preliminare include una bozza di "Ubiquitous Language" (UL), che rappresenta il
linguaggio condiviso all'interno del team di progetto. Questo linguaggio è cruciale per garantire una
comunicazione precisa e una comprensione comune del dominio. Inoltre, è iniziata la definizione dei casi
d'uso e sono state registrate le riunioni per riferimento futuro.\\

\subsubsection{Ubiquitous Language (UL) e casi d'uso}
L'Ubiquitous Language (UL) è un elemento fondamentale nella fase di Scoping and Initiation. Non si
tratta solo di un elenco di parole, ma di un insieme di termini e concetti specifici del dominio.
Questo linguaggio condiviso è stato documentato nel rapporto sul \textbf{Domain Driven Development}
(DDD) e servirà da guida per garantire una comunicazione precisa tra i membri del team e per
l'implementazione del codice.\\

\subsubsection{Conditions of Satisfaction (COS)}
Dalla riunione preliminare, sono state identificate le "Conditions of Satisfaction" (COS) che rappresentano
le aspettative chiave del cliente e del progetto. Queste includono vincoli di budget, vincoli di tempo,
soddisfazione del cliente, rispetto del piano di qualità, documentazione completa dell'architettura
software e compatibilità con l'architettura esistente.\\

La comprensione chiara di queste COS è essenziale per garantire che il progetto soddisfi gli obiettivi
aziendali e del cliente.\\

\subsection{Primo project scoping meeting}
Il primo incontro di project scoping ha coinvolto una vasta gamma di partecipanti, compresi un business manager, un project manager, programmatori senior, un tecnografo, stakeholder ed esperti del dominio. In questa fase, si è cercato di definire le principali sfide e opportunità del progetto.

Poiché il progetto ScaFi si basa su concetti esistenti, è stato fondamentale acquisire una comprensione approfondita del dominio. Inizialmente, non è stata stabilita un'agenda rigida, ma sono stati richiesti agli esperti del dominio di fornire documentazione che il team avrebbe dovuto esaminare prima del prossimo incontro. Questo approccio ha permesso di fare domande più mirate nella fase successiva.

Durante questa riunione, il team ha presentato alcune possibili soluzioni alle sfide identificate, confrontandole poi con gli esperti del dominio. Le Conditions of Satisfaction (COS) sono state discusse ed aggiornate in base alle nuove informazioni.

\subsubsection{Project Overview Statement (POS) e Analisi dei rischi}
Successivamente al primo meeting, è stato redatto il documento "Project Overview Statement" e condotta un'analisi dei rischi per comprendere meglio il contesto e le sfide del progetto.

\subsection{Secondo Project Scoping Meeting}
Nel secondo incontro di project scoping, sono stati nuovamente coinvolti un business manager, un project manager, programmatori senior, un tecnografo e due esperti del dominio. L'obiettivo era rifinire ulteriormente i requisiti del progetto.

Inizialmente, è stato richiesto agli esperti del dominio di rivedere i documenti esistenti per garantire la correttezza delle definizioni e dei concetti. Successivamente, sono state poste ulteriori domande per chiarire gli aspetti cruciali delle funzionalità di ScaFi.

Durante questo incontro, è iniziata la sviluppo di una bozza della "Requirement Breakdown Structure" (RBS), che rappresenterà una mappa dettagliata dei requisiti del progetto.

\subsubsection{Requirement Breakdown Structure (RBS)}
La Requirement Breakdown Structure (RBS) è un componente chiave nella definizione dei requisiti del progetto. Questa struttura dettagliata dei requisiti sarà completata successivamente dal team e presentata agli esperti del dominio e agli stakeholder al prossimo incontro.

\subsection{Terzo Project Scoping Meeting}
Nel terzo incontro di project scoping, sono stati coinvolti un project manager, tre membri chiave del team e due esperti del dominio. L'obiettivo principale era confermare la Requirement Breakdown Structure (RBS) e definire il modello di ciclo di vita del progetto (PMLC) da adottare.

\subsubsection{Modello PMLC}
Il modello di ciclo di vita del progetto scelto rientra nell'ambito dell'Agile Project Management ed è noto come "Adaptive Project Management Life Cycle Model." Questo modello offre flessibilità quando l'obiettivo è noto ma la soluzione è influenzata da cambiamenti previsti. Questo approccio si adatta alle sfide dinamiche che possono emergere durante il progetto.

\subsubsection{Workflow}
Durante questo incontro, sono stati identificati processi che possono essere automatizzati all'interno del progetto. Ciò include il testing, l'aggiornamento delle dipendenze, il rilascio semantico e l'auto-merging dopo aver superato controlli automatici.

In sintesi, la fase di Scoping and Initiation è cruciale per stabilire una base solida per il progetto di espansione di ScaFi. Attraverso incontri, analisi dei requisiti e definizione del modello di ciclo di vita, il team ha guadagnato una comprensione chiara del progetto e delle aspettative degli stakeholder. Questo prepara il terreno per la fase successiva, in cui si inizierà a pianificare e sviluppare il progetto in modo coerente con gli obiettivi stabiliti finora.
