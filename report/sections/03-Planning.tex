\section{Planning}

%identifica le attività che devono essere svolte per implementare i requisiti e completare il progetto, stima il tempo, i costi e le risorse necessarie e ottiene l'approvazione del piano

\subsection{Durata prevista}
Il progetto è stato classificato come medio-grande, con durata massima del planning individuata come 2 giorni.
È possibile eccedere di 1 giorno al massimo, nel caso in cui si riscontrino problemi organizzativi o altro.

In tutte le riunioni organizzate, i partecipanti saranno i seguenti:
\begin{itemize}
    \item 1 project manager;
    \item 4 componenti del team, tra cui 1 tecnografo
\end{itemize}

Le riunioni saranno tenute in una stanza apposita per i meeting dotata di lavagna, oppure in forma
telematica in caso non si sia riusciti ad organizzare un incontro fisico per la distanza tra i membri del team.

\subsection{Prima Joint Project Planning Session}
È stato inizialmente fatto un rapido riassunto della riunione precedente, concentrandosi sui documenti stilati e su ciò che
è stato detto e consigliato dagli esperti del dominio.

L'obiettivo primario di questo incontro è la creazione di una bozza della \textbf{"Work Breakdown Structure"} (WBS), ovvero la scomposizione
del progetto in attività, prioritizzando le attività più importanti e definendo le dipendenze tra di esse.

Per garantire una corretta indvididuazione dei task, il Project Manager ha suddiviso il team di sviluppo in due sottogruppi.

Ogni sottogruppo si occupava di invididuare i task principali da svolgere all'interno del sottoprogetto assegnato, e di
definire le dipendenze tra di essi.

In questo modo è stato possibile lavorare parallelamente, e in modo più efficiente.

\subsubsection{Approccio di progetto concordato}
Il lavoro è stato suddiviso in sprint settimanali, in modo da poter avere un feedback costante sullo stato di avanzamento
del progetto.

Gli strumenti utilizzati per l'organizzazione del lavoro sono stati i Projects e le issue di GitHub, in modo da poter
avere un'idea chiara di quali task sono stati assegnati e quali sono ancora da assegnare ed il loro stato di avanzamento.


\subsection{Seconda Joint Project Planning Session}


\subsubsection{Work Breakdown Structure}
\subsubsection{Project Network Diagram}

\subsection{Terza Joint Project Planning Session}
\subsubsection{Project Network Diagram (aggiornato con la durata delle attività)}
\subsubsection{Stime delle risorse}
\subsubsection{Stime dei tempi}

\subsection{Project Proposal}

