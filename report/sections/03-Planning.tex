\section{Planning}

La fase di pianificazione in un progetto rappresenta un passo cruciale per garantire il suo successo. 
Durante questa fase, vengono identificate le attività necessarie per implementare i requisiti del progetto, vengono stimate le tempistiche, 
i costi e le risorse necessarie, e si ottiene l'approvazione del piano da parte delle parti interessate.

\subsection{Durata prevista}
Il progetto è stato categorizzato come di dimensioni medio-grandi, con una durata massima di pianificazione fissata a 2 giorni. 
Tuttavia, è stata prevista la possibilità di un'estensione di un massimo di 1 giorno, nel caso in cui si verifichino problemi 
organizzativi o altre circostanze eccezionali.

Nelle riunioni pianificate per questa fase, i partecipanti saranno i seguenti:
\begin{itemize}
    \item 1 project manager;
    \item 4 membri del team, inclusi 1 tecnografo.
\end{itemize}

Le riunioni sono state pianificate in una stanza appositamente dedicata ai meeting, dotata di una lavagna, ma sono state organizzate 
anche in forma telematica quando non è stato possibile organizzare incontri fisici a causa della distanza tra i membri del team.

\subsection{Prima Joint Project Planning Session}
Durante la prima sessione congiunta di pianificazione del progetto, è stata effettuata una sintesi delle discussioni precedenti, 
concentrandosi sui documenti creati e sui suggerimenti forniti dagli esperti del dominio.

L'obiettivo principale di questa sessione è stata la creazione della "Work Breakdown Structure" (WBS), ovvero la suddivisione 
del progetto in attività. Le attività sono state priorizzate, e sono state definite le dipendenze tra di esse.

Per garantire una corretta identificazione delle attività, il project manager ha suddiviso il team di sviluppo in due sottogruppi. 
Ciascun sottogruppo aveva il compito di individuare le attività principali all'interno del sotto-progetto assegnato e di definire 
le dipendenze tra di esse. Questo approccio ha consentito di lavorare in modo parallelo ed efficiente.

\subsubsection{Approccio di progetto concordato}
Il lavoro è stato organizzato in sprint settimanali, garantendo così un feedback continuo sullo stato di avanzamento del progetto. 
Per la gestione delle attività, sono stati utilizzati gli strumenti Projects e le issue di GitHub. Questi strumenti hanno fornito 
una visione chiara dei task assegnati e di quelli ancora da assegnare, consentendo di monitorare lo stato di avanzamento.

\subsubsection{Work Breakdown Structure}
La Work Breakdown Structure (WBS) completa è disponibile nel documento allegato.

\subsection{Seconda Joint Project Planning Session}
Durante la seconda sessione congiunta di pianificazione del progetto, è stata pianificata la condivisione del piano con il committente. 
Questo passaggio è fondamentale per ottenere un feedback sul lavoro svolto e sulle scelte effettuate.

Sono stati confermati i task individuati nella prima riunione, e si è proceduto con la stima delle risorse e dei tempi necessari per 
ciascuna attività, utilizzando la Delphi Technique. In questa fase, solo il team di lavoro interno ha partecipato attivamente.

\subsubsection{Stime delle risorse}
Di seguito sono riportate le stime delle risorse umane necessarie per il progetto:

\begin{itemize}
    \item Project Manager: coordinatore del progetto responsabile della pianificazione, esecuzione e controllo delle attività, 
    comunicazione con gli stakeholder e risoluzione dei problemi.
    \item Team di sviluppo senior: esperti con competenze tecniche avanzate, responsabili delle soluzioni complesse e della guida del team.
    \item Team di sviluppo junior: membri meno esperti, responsabili dell'implementazione di componenti più semplici e con 
    l'obiettivo di crescita professionale.
    \item Sistemista: professionista specializzato nella gestione e manutenzione dei sistemi informatici, responsabile dell'infrastruttura 
    tecnologica del progetto.
\end{itemize}

\subsubsection{Stime dei tempi}
Come precedentemente menzionato, le attività sono state organizzate in sprint settimanali, considerando le relazioni tra i compiti 
per evitare ritardi. Le stime temporali sono indicative, e possono essere soggette a modifiche. La pianificazione è iniziata con 
l'allocazione delle ore del primo sprint per tutti i sotto-progetti, con una priorità basata sulla complessità delle attività e 
sulla risoluzione delle dipendenze tra di esse.

Durante la pianificazione, si è prestata attenzione a bilanciare le ore settimanali dedicate a ciascun sotto-progetto, 
al fine di evitare ritardi dovuti a dipendenze tra le attività. Le risorse sono state riassegnate o deallocate durante gli sprint 
per ottimizzare i costi senza compromettere le prestazioni.
