\section{Planning}

%identifica le attività che devono essere svolte per implementare i requisiti e completare il progetto, stima il tempo, i costi e le risorse necessarie e ottiene l'approvazione del piano

\subsection{Durata prevista}
Il progetto è stato classificato come medio-grande, con durata massima del planning individuata come 2 giorni.
È possibile eccedere di 1 giorno al massimo, nel caso in cui si riscontrino problemi organizzativi o altro.

In tutte le riunioni organizzate, i partecipanti saranno i seguenti:
\begin{itemize}
    \item 1 project manager;
    \item 4 componenti del team, tra cui 1 tecnografo
\end{itemize}

Le riunioni saranno tenute in una stanza apposita per i meeting dotata di lavagna, oppure in forma
telematica in caso non si sia riusciti ad organizzare un incontro fisico per la distanza tra i membri del team.

\subsection{Prima Joint Project Planning Session}
È stato inizialmente fatto un rapido riassunto della riunione precedente, concentrandosi sui documenti stilati e su ciò che
è stato detto e consigliato dagli esperti del dominio.

L'obiettivo primario di questo incontro è la creazione della \textbf{"Work Breakdown Structure"} (WBS), ovvero la scomposizione
del progetto in attività, prioritizzando le attività più importanti e definendo le dipendenze tra di esse.

Per garantire una corretta indvididuazione dei task, il Project Manager ha suddiviso il team di sviluppo in due sottogruppi.

Ogni sottogruppo si occupava di invididuare i task principali da svolgere all'interno del sottoprogetto assegnato, e di
definire le dipendenze tra di essi.

In questo modo è stato possibile lavorare parallelamente, e in modo più efficiente.

\subsubsection{Approccio di progetto concordato}
Il lavoro è stato suddiviso in sprint settimanali, in modo da poter avere un feedback costante sullo stato di avanzamento
del progetto.

Gli strumenti utilizzati per l'organizzazione del lavoro sono stati i Projects e le issue di GitHub, in modo da poter
avere un'idea chiara di quali task sono stati assegnati e quali sono ancora da assegnare ed il loro stato di avanzamento.

\subsubsection{Work Breakdown Structure}
TODO: inserire WBS
\subsubsection{Project Network Diagram}
TODO: inserire PND

\subsection{Seconda Joint Project Planning Session}
In questa ultima riunione, si è prevista la condivisione con il committente del planning del progetto, in modo da
poter avere un feedback sul lavoro svolto e sulle scelte fatte.

Confermati i task individuati nella prima riunione, si è proceduto con la stima delle risorse e dei tempi necessari
per lo svolgimento di ogni attività, procedendo con la Delphi Technique. In questa fase ha partecipato attivamente
solo il team di lavoro interno.

\subsubsection{Stime delle risorse}
In seguito sono riportate le stime delle risorse, lato personale, necessarie per lo svolgimento del progetto.

\begin{itemize}
    \item Project Manager: Coordinatore del progetto responsabile della pianificazione, esecuzione e controllo delle attività, comunicazione con gli stakeholder e risoluzione dei problemi.
    \item Team di sviluppo senior: Esperti con competenze tecniche avanzate, responsabili delle soluzioni complesse e della guida del team.
    \item Team di sviluppo junior: Membri meno esperti, responsabili dell'implementazione di componenti più semplici e con l'obiettivo di crescita professionale.
    \item Sistemista: Professionista specializzato nella gestione e manutenzione dei sistemi informatici, responsabile dell'infrastruttura tecnologica del progetto.
\end{itemize}

\subsubsection{Stime dei tempi}
Come accennato in precedenza, abbiamo scelto di organizzare i compiti all'interno di sprint settimanali, considerando le relazioni tra i
compiti e assicurandoci di concentrare le risorse in modo che lo sviluppo di un sotto-progetto non influisca negativamente sull'avanzamento
di un altro. Le stime fornite sono indicative, e vi è la possibilità di modificare la durata e il numero degli sprint se necessario.

Abbiamo iniziato allocando le ore del primo sprint per tutti i sotto-progetti, iniziando dai compiti indipendenti tra loro. Questa allocazione
è stata basata sulle stime di tempo per ciascun compito, risorse disponibili e dando priorità ai compiti che risolvono le dipendenze tra i
sotto-progetti. Abbiamo ripetuto questo processo fino a completare tutti i compiti.

Una volta completata la prima distribuzione dei compiti, abbiamo cercato di bilanciare le ore settimanali dedicate a ciascun sotto-progetto,
evitando situazioni in cui i ritardi dovuti alle dipendenze potrebbero verificarsi. Le risorse sono state riassegnate o deallocate durante gli
sprint per ottimizzare i costi senza compromettere le prestazioni.


