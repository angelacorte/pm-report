\section{Monitoring and Controlling}
Con lo scopo di garantire il rispetto delle deadline e la qualità del prodotto finale, è stato necessario adottare un sistema di monitoraggio e controllo del progetto.\\

Il team di sviluppo deve attenersi alle regole operative individuate ed il Project Manager deve attenersi al sistema di reporting.

Si vuole rimarcare l'importanza dei meeting giornalieri e settimanali, che permettono di monitorare lo stato di avanzamento del progetto e consentono agli sviluppatori di scambiarsi eventuali consigli implementativi.\\


% \subsection{Sistema di Reporting}%%

\subsection{Sprint Review}
Con cadenza settimanale, il team si riunisce al completo per discutere lo stato di avanzamento del progetto.\\

Viene aggiornato il Product Backlog segnalando le attività che sono state completate, quelle che sono in corso e quelle che possibilmente dovranno essere completate entro la sprint review successiva.\\

Mentre a cadenza bisettimanale verranno aggiornati gli stakeholders e verranno informati anche gli esperti del dominio, che potranno fornire un feedback sul lavoro svolto.\\

\subsection{GitHub Project Board}
Il team di sviluppo utilizza GitHub Project Board per tenere traccia delle varie attività.\\

Il Project Board è suddiviso in tre colonne:
\begin{itemize}
    \item \textbf{To Do}: contiene le attività che devono essere svolte;
    \item \textbf{In Progress}: contiene le attività che sono in corso di svolgimento;
    \item \textbf{Done}: contiene le attività che sono state completate.
\end{itemize}

Una funzionalità molto utile di GitHub Project Board è la possibilità di assegnare un'attività ad un membro del team, in modo da tenere traccia di chi sta svolgendo cosa.\\

\subsection{Issues Log}
È stato deciso di utilizzare GitHub come servizio di Distributed Version Control System (DVCS), che offre anche un mezzo altamente efficiente per gestire
e risolvere le problematiche emerse durante lo sviluppo del progetto. 

Gli sviluppatori hanno la possibilità di segnalare e documentare
le problematiche direttamente sulla piattaforma GitHub in modo accurato e tracciabile. Questo processo consente a un altro membro del
team incaricato di affrontare l'attività di comprendere e replicare l'errore prima di risolverlo.

Inoltre, GitHub include una funzionalità
di chiusura automatica delle issue una volta che la relativa soluzione è stata implementata, garantendo così una gestione efficiente delle problematiche.

In aggiunta, la GitHub Project Board offre la possibilità di aprire e monitorare le issues in modo organizzato, consentendo al team di tenere traccia
dello stato di ogni problema e delle relative soluzioni in modo chiaro e strutturato.

