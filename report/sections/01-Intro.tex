\section{Introduzione}

\textit{Rustfields} è un progetto nato dalla necessità di affrontare le sfide emergenti legate alla gestione delle reti distribuite e alla
coordinazione di dispositivi con risorse limitate. \\

In un mondo sempre più connesso, la gestione di reti composte da dispositivi eterogenei è diventata la norma.
Dalla gestione del traffico urbano alla raccolta dati industriale, la necessità di far collaborare queste entità rappresenta una sfida significativa.\\

Uno dei principali problemi affrontati da \textit{Rustfields} è l'esecuzione di algoritmi complessi su dispositivi con risorse hardware limitate.
Questi dispositivi, noti come "thin devices", richiedono una gestione efficiente delle risorse per garantire prestazioni ottimali. \\

\textit{Rustfields} si basa sul paradigma dell'Aggregate Computing, una metodologia di programmazione che semplifica la gestione delle reti distribuite.\\

Il cuore di \textit{Rustfields} è il "field calculus", una tecnica che consente di coordinare le interazioni tra dispositivi in modo efficiente ed elegante.
Il framework è stato sviluppato estendendo ScaFi, una libreria e framework di programmazione aggregata basato su Scala.\\

L'obiettivo principale di \textit{Rustfields} è rendere il "field calculus" accessibile e utilizzabile anche su thin devices.
Questo approccio apre la strada a nuove possibilità nell'ambito delle reti distribuite.\\

In un mondo in continua evoluzione, la gestione efficiente delle reti distribuite è diventata fondamentale.
\textit{Rustfields} vuole offrire una soluzione innovativa che semplifica lo sviluppo di applicazioni distribuite, garantendo al contempo una
gestione efficiente delle risorse limitate dei dispositivi. Il progetto è una risposta alle esigenze di un futuro sempre più
connesso e intelligente, in cui le reti distribuite svolgono un ruolo cruciale. Con \textit{Rustfields}, le sfide diventano opportunità e
la complessità delle reti distribuite viene affrontata in modo efficiente ed efficace.