\section{Introduzione}

\textit{Rustfields} è un progetto nato dalla volontà di estendere ScaFi: un framework scritto in Scala che permette di effettuare
Aggregate Programming \\

In tanti contesti, ci si trova a dover gestire una grande quantità di dispositivi eterogeneei che devono collaborare tra loro, nel
raggiungimento di un obiettivo. Queste reti di dispositivi vengono definite Collective Adapative Systems (CAS). \\

Il punto focale di questi CAS è che non è presente un entità centrale che coordina i dispositivi, ma ogni dispositivo è autonomo e
deve collaborare con gli altri, adattandosi ai cambiamenti dell'ambiente. \\

Il paradigma dell'Aggregate Programming si propone di semplificare la gestione di queste reti di dispositivi, trattando tutta la 
rete di dispositivi come un'unica entità. \\

L'Aggregate Programming si basa sul Field Calculus, un sistema formale sviluppato per analizzare formalmente i Field e il loro calcolo.

Uno dei principali problemi affrontati in questo progetto è l'esecuzione di programmi aggregati su dispositivi con risorse hardware 
limitate. Questi dispositivi, noti come "thin devices", richiedono una gestione efficiente delle risorse per garantire prestazioni ottimali. \\

In un mondo sempre più connesso, si stanno sempre più diffondendo situazioni in cui è necessario
la gestione di reti composte da dispositivi eterogenei è diventata la norma.
Dalla gestione del traffico urbano alla raccolta dati industriale, la necessità di far collaborare queste entità rappresenta una sfida 
significativa. \\

L'obiettivo principale di \textit{Rustfields} è quindi quello di rendere il "field calculus" accessibile e utilizzabile anche su thin devices.