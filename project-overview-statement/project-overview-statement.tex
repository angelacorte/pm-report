
\documentclass[12pt, a4paper]{article}
\usepackage[utf8]{inputenc}
\usepackage{fontenc}
\usepackage{xcolor}
\usepackage{hyperref}
\usepackage[english]{babel}
\usepackage[inline]{enumitem}
\usepackage{graphicx}
\usepackage{cleveref}

\graphicspath{ {res/} }

\newcommand{\versionmajor}{0}
\newcommand{\versionminor}{1}
\newcommand{\versionpatch}{0}
\newcommand{\version}{\versionmajor.\versionminor.\versionpatch}

\title{\LARGE
    Rustfields \\ 
    \small
    Project Overview Statement
    }

\author{
    Angela Cortecchia \\ 
    \small 
    angela.cortecchia@studio.unibo.it
    \and
    Paolo Penazzi \\ 
    \small
    paolo.penazzi@studio.unibo.it
}

\date{\small }

\begin{document}
\maketitle
\par\noindent\rule{\textwidth}{0.5pt}


\section*{Problem/Opportunity}
L'aggregate computing è una disciplina informatica che si concentra sulla manipolazione e
l'elaborazione di dati distribuiti su una rete di dispositivi. Questa tecnologia può essere utilizzata in una vasta gamma
di applicazioni e contesti per risolvere problemi complessi.

Tuttavia, il campo di applicazione dell'aggregate computing attualmente ha come vincolo quello di essere applicabile a dispositivi dotati di JVM (Java Virtual Machine),
ciò comporta una notevole limitazione nella tipologia di dispositivi utilizzabili.

Con l'evoluzione delle reti distrubuite, si è dunque giunti alla necessità di espandere le tipologie di dispositivi su cui è possibile effettuare
questo tipo di calcoli.


\section*{Project Goal}
L'obiettivo principale di Rustfields è chiaro e centrato: consentire l'esecuzione affidabile del field calculus su una gamma diversificata di dispositivi.

Lo scopo deve essere raggiunto garantendo efficienza, robustezza e adattabilità nelle condizioni variabili delle reti distribuite.


\section*{Project Objectives}
Gli obiettivi di questo progetto sono:
\begin{itemize}
    \item Aggiornare la versione esistente di ScaFi (Scala Fields), una libreria che offre funzionalità
          per l'aggregate computing, dalla versione di Scala 2 a Scala 3
    \item Sviluppare un framework di programmazione distribuita che consenta l'esecuzione di funzionalità di aggregate computing su dispositivi
          con risorse limitate, eliminando la necessità di dipendere dalla Java Virtual Machine (JVM) e fornendo un'esperienza fluida su thin devices
    \item Aggiungere i "campi" o "fields" come entità principali nel calcolo
    \item Garantire che il framework Rustfields sia in grado di scalare in modo flessibile per supportare reti di dispositivi di dimensioni
          variabili, da piccoli sistemi locali a reti distribuite di grandi dimensioni
    \item Ampliare la test suite attualmente esistente in maniera tale da renderla più completa e comprensibile.
    \item Garantire robustezza e scalabilità del nuovo framework
\end{itemize}

\section*{Success Criteria}
\begin{itemize}
    \item Porting completo del core di ScaFi da Scala 2 a Scala 3
    \item Integrazione dei "fields" come entità principali nel calcolo, consentendo operazioni avanzate di aggregate computing
    \item Porting del core di ScaFi nel linguaggio di programmazione Rust ed integrazione per renderlo interoperabile con ScaFi
\end{itemize}

\section*{Assumptions}
\begin{itemize}
    \item Si presuppone che il team lavori a tempo pieno
    \item Sarà garantita la presenza di esperti del dominio all'interno del team
\end{itemize}

\section*{Risks}
Presente un documento dell'analisi dei rischi denominato \textbf{"Risk Analysis"}.

\section*{Obstacles}
\begin{itemize}
    \item La creazione di un framework di programmazione distribuita per l'aggregate computing su
          thin devices richiede la gestione di complessità tecnica significativa
    \item L'aggiornamento da Scala 2 a Scala 3 potrebbe comportare sfide di compatibilità che richiedono
          la revisione e la modifica del codice esistente.
    \item Potrebbero subentrare problemi di compatibilità tra Rust e Scala
    \item Essendo un ambito inesplorato, bisogna considerare che la documentazione inerente potrebbe scarseggiare

\end{itemize}

\end{document}