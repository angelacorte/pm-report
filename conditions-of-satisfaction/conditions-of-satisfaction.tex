\documentclass[12pt, a4paper]{article}
\usepackage[utf8]{inputenc}
\usepackage{fontenc}
\usepackage{xcolor}
\usepackage{hyperref}
\usepackage[english]{babel}
\usepackage[inline]{enumitem}
\usepackage{graphicx}
\usepackage{cleveref}

\newcommand{\versionmajor}{0}
\newcommand{\versionminor}{1}
\newcommand{\versionpatch}{0}
\newcommand{\version}{\versionmajor.\versionminor.\versionpatch}

\begin{document}

\section{Conditions of Satisfaction}

Dalle richieste del committente, emerse durante il primo meeting, il team ha individuato le seguenti \textbf{Condition of 
Satisfaction}:

\begin{enumerate}
    \item \textbf{Vincoli di Budget:} Il progetto deve essere portato a termine rientrando nel budget previsto per la 
    sua esecuzione.
    \item \textbf{Vincoli di tempo:} Portare a termine il progetto rientrando nelle tempistiche previste.
    \item \textbf{Soddisfazione del cliente:} Il prodotto deve soddisfare il cliente, in modo tale che egli possa 
    utilizzarlo come base per sviluppi futuri.
    \item \textbf{Rispetto del piano di qualità:} Rispetto delle convenzioni adottate dalle varie tecnologie utilizzate,
    rispetto degli standard concordati durante tutte le fasi del progetto.
    \item \textbf{Documentazione completa dell'architettura software:} La documentazione deve essere chiara e consultabile
    nel caso in cui siano necessari chiarimenti.
    \item \textbf{Compatibilità:} Il prodotto andrà a rimpiazzare il cuore di un'architettura composta da più moduli.
    È fondamentale che possa sostituire il prodotto già esistente senza causare problemi di compatibilità.
    \item \textbf{}
    \item \textbf{}
    \item \textbf{}
\end{enumerate}

\end{document}