\documentclass[12pt, a4paper]{article}
\usepackage[utf8]{inputenc}
\usepackage{fontenc}
\usepackage{xcolor}
\usepackage{hyperref}
\usepackage[english]{babel}
\usepackage[inline]{enumitem}
\usepackage{graphicx}
\usepackage{cleveref}
\usepackage{enumitem}

\setlist[enumerate,1]{label=\arabic*}
\setlist[enumerate,2]{label=\theenumi.\arabic*}
\setlist[enumerate,3]{label=\theenumii.\arabic*}
\setlist[enumerate,4]{label=\theenumiii.\arabic*}

\graphicspath{ {res/} }

\newcommand{\versionmajor}{0}
\newcommand{\versionminor}{1}
\newcommand{\versionpatch}{0}
\newcommand{\version}{\versionmajor.\versionminor.\versionpatch}

\title{\LARGE
    Rustfields \\ 
    \small
    Work Breakdown Structure
    }

\author{
    Angela Cortecchia \\ 
    \small 
    angela.cortecchia@studio.unibo.it
    \and
    Paolo Penazzi \\ 
    \small
    paolo.penazzi@studio.unibo.it
}

\date{\small }

\begin{document}
\maketitle
\newpage


\section*{Work Breakdown Structure}

Il team ha delineato la seguente Work Breakdown Structure, riportata di seguito anche con una
rappresentazione grafica.

\subsection*{Goal}

Il goal principale del progetto, individuato dal team è \textbf{estendere ScaFi } una piattaforma che permette
di scrivere programmi aggregati per reti di dispositivi.

\subsection*{Requirements}

A partire dal goal principale del progetto, sono stati individuati diversi requisiti:

\begin{enumerate}
    \color{teal}
    \item \textbf{Aggregate Computing nativo}
          \begin{enumerate}
              \item \textbf{Esecuzione di un programma aggregato}
                    \begin{enumerate}
                        \item \textbf{Esplorare altri paradigmi di esecuzione di un programma aggregato}
                        \item \textbf{Implementare il concetto di Slot}
                        \item \textbf{Implementare il concetto di Path}
                        \item \textbf{Implementare il concetto di Export}
                        \item \textbf{Implementare il concetto di Context}
                        \item \textbf{Valutare di rendere immutabile la RoundVM}
                        \item \textbf{Implementare il concetto di stato della VM}
                        \item \textbf{Implementare la RoundVM}
                    \end{enumerate}

              \item \textbf{Scrittura di un programma aggregato}
                    \begin{enumerate}
                        \item \textbf{Valutare l'introduzione di altri costrutti come Exchange e Share}
                        \item \textbf{Implementare i costrutti del linguaggio attualmente in ScaFi}
                              \begin{enumerate}
                                  \item \textbf{Implementare il costrutto Rep}
                                  \item \textbf{Implementare il costrutto Nbr}
                                  \item \textbf{Implementare il costrutto Foldhood}
                                  \item \textbf{Implementare il costrutto Branch}
                              \end{enumerate}
                    \end{enumerate}


              \item \textbf{Test}
                    \begin{enumerate}
                        \item \textbf{Testare l'esecuzione dei round}
                        \item \textbf{Testare la semantica del linguaggio}
                        \item \textbf{Trovare nuovi modi di testare il software}
                    \end{enumerate}
              \item \textbf{Configurare il workflow di release semantica}

          \end{enumerate}

          \color{cyan}
    \item \textbf{Aggiornamento di ScaFi}
          \begin{enumerate}
              \item \textbf{Adottare Scala 3}
                    \begin{enumerate}
                        \item \textbf{Aggiornare i concetti per l'esecuzione di un programma aggregato}
                              \begin{enumerate}
                                  \item \textbf{Implementare Slot in Scala 3}
                                  \item \textbf{Implementare Path in Scala 3}
                                  \item \textbf{Implementare Export in Scala 3}
                                  \item \textbf{Implementare Context in Scala 3}
                                  \item \textbf{Implementare VMStatus in Scala 3}
                                  \item \textbf{Implementare RoundVM in Scala 3}
                              \end{enumerate}

                        \item \textbf{Aggiornare il linguaggio in Scala 3}
                              \begin{enumerate}
                                  \item \textbf{Implementare Rep in Scala 3}
                                  \item \textbf{Implementare Nbr in Scala 3}
                                  \item \textbf{Implementare Branch in Scala 3}
                                  \item \textbf{Implementare Foldhood in Scala 3}
                                  \item \textbf{Implementare i Builtins in Scala 3}
                              \end{enumerate}

                    \end{enumerate}
              \item \textbf{Espandere la suite di test}
                    \color{blue}
              \item \textbf{Abilitare l'utilizzo di Fields reificati}
                    \begin{enumerate}
                        \item \textbf{Creare i Field come entità di primo livello}
                        \item \textbf{Adattare i costrutti del linguaggio per essere utilizzzabili sui Fields}
                        \item \textbf{Rendere le operazioni sui field utilizzabili come monadi}
                        \item \textbf{Creare una libreria per l'utilizzo dei field reificati}
                    \end{enumerate}
              \item \textbf{Configurare il workflow di release semantica}

          \end{enumerate}

          \color{magenta}
    \item \textbf{Interoperabilità tra le due versioni:}
          \begin{enumerate}
              \item \textbf{Standardizzare il formato dei messaggi.}
              \item \textbf{Implementare un wrapper per il core in Rust.}
              \item \textbf{Definire un progetto SBT e Cargo.}
          \end{enumerate}
\end{enumerate}

\end{document}