
\documentclass[12pt, a4paper]{article}
\usepackage[utf8]{inputenc}
\usepackage{fontenc}
\usepackage{xcolor}
\usepackage{hyperref}
\usepackage[english]{babel}
\usepackage[inline]{enumitem}
\usepackage{graphicx}
\usepackage{cleveref}

\graphicspath{ {res/} }

\newcommand{\versionmajor}{0}
\newcommand{\versionminor}{1}
\newcommand{\versionpatch}{0}
\newcommand{\version}{\versionmajor.\versionminor.\versionpatch}

\title{\LARGE
    Rustfields \\ 
    \small
    Sprints
    }

\author{
    Angela Cortecchia \\ 
    \small 
    angela.cortecchia@studio.unibo.it
    \and
    Paolo Penazzi \\ 
    \small
    paolo.penazzi@studio.unibo.it
}

\date{\small }

\begin{document}
\maketitle
\newpage

\section*{Sprints}

\par\noindent\rule{\textwidth}{0.5pt}

\subsection*{Sprint 1}

\begin{itemize}
    \item Goals:
          \begin{itemize}
              \color{teal}
              \item In RuFi-core
                    \begin{itemize}
                        \item Implementare il concetto di VM Status
                    \end{itemize}
                    \color{cyan}
              \item In ScaFi-core
                    \begin{itemize}
                        \item Definire un API di alto livello
                    \end{itemize}
          \end{itemize}
    \item Review:
          \begin{itemize}
              \color{teal}
              \item In RuFi-core
                    \begin{itemize}
                        \item 1.1.2 - Implementato il concetto di Slot (4 ore)
                        \item 1.1.3 - Implementato il concetto di Path (4 ore)
                        \item 1.1.4 - Implementato il concetto di Export (4 ore)
                        \item 1.4 - Configurato il workflow di release semantica (6 ore)
                    \end{itemize}
                    \color{cyan}
              \item In ScaFi-core
                    \begin{itemize}
                        \item 2.1.1.2 - Implementato il concetto di Path (3 ore)
                        \item 2.1.1.1 - Implementato il concetto Slot (3 ore)
                    \end{itemize}
                    \color{blue}
              \item In ScaFi-fields
                    \begin{itemize}
                        \item 2.3.1 - Iniziato il lavoro inerente ai fields reificati (3.5 ore)
                    \end{itemize}
          \end{itemize}
\end{itemize}

\par\noindent\rule{\textwidth}{0.5pt}

\subsection*{Sprint 2}

\begin{itemize}
    \item Goals:
          \begin{itemize}
              \color{teal}
              \item In RuFi-core
                    \begin{itemize}
                        \item Implementare il Context
                        \item Iniziare l'implementazione del concetto di RoundVM
                    \end{itemize}
                    \color{cyan}
              \item In ScaFi-core
                    \begin{itemize}
                        \item Implementare il concetto di Export
                        \item Reificare i fields
                        \item Implementare RoundVM
                        \item Implementare un API per i fields reificati
                        \item Configurare release semantica
                    \end{itemize}
          \end{itemize}
    \item Review:
          \begin{itemize}
              \color{teal}
              \item In RuFi-core
                    \begin{itemize}
                        \item 1.1.5 - Implementato il Context (3 ore)
                        \item 1.1.8 - Iniziata l'implementazione della RoundVM (4 ore)
                        \item 1.1.3 - Riparato il comportamento di Path (2 ore)
                        \item 1.1.5 - Refattorizzata l'API di Context (2 ore)
                    \end{itemize}
                    \color{cyan}
              \item In ScaFi-core
                    \begin{itemize}
                        \item 2.4 - Subentrati problemi nella configurazione di release semantica (4 ore)
                        \item 2.3.1 - Implementati i fields reificati (5 ore)
                        \item 2.1.1.3 - Implementato il concetto di Export (3 ore)
                        \item 2.1.1.4 - Implementato il concetto di Context (4 ore)
                    \end{itemize}
          \end{itemize}
\end{itemize}

\par\noindent\rule{\textwidth}{0.5pt}

\subsection*{Sprint 3}

\begin{itemize}
    \item Goals:
          \begin{itemize}
              \color{teal}
              \item In RuFi-core
                    \begin{itemize}
                        \item Completare la RoundVM
                    \end{itemize}
                    \color{cyan}
              \item In ScaFi-core
                    \begin{itemize}
                        \item Iniziare ad implementare i costrutti del field calculus
                        \item Finire di implementare il concetto di RoundVM
                    \end{itemize}
          \end{itemize}
    \item Review:
          \begin{itemize}
              \color{teal}
              \item In RuFi-core
                    \begin{itemize}
                        \item 1.1.8 - Implementata la RoundVM (3.5 ore)
                    \end{itemize}
                    \color{cyan}
              \item In ScaFi-core
                    \begin{itemize}
                        \item 2.1.2.2 - Implementato il costrutto del field calculus \texttt{nbr} (3 ore)
                        \item 2.1.2.1 - Implementato il costrutto del field calculus \texttt{rep} (3 ore)
                        \item 2.1.1.6 - Deciso di fermare momentaneamente lo sviluppo di RoundVM per provare ad utilizzare quella sviluppata in RuFi-core (3 ore)
                    \end{itemize}
          \end{itemize}
\end{itemize}

\par\noindent\rule{\textwidth}{0.5pt}

\subsection*{Sprint 4}

\begin{itemize}
    \item Goals:
          \begin{itemize}
              \color{teal}
              \item In RuFi-core
                    \begin{itemize}
                        \item Impostare il progetto Java-Rust
                        \item Sperimentare diverse soluzioni per la comunicazione tra Java e Rust (JNI, Wasmer, ScalaNative)
                    \end{itemize}
                    \color{cyan}
              \item In ScaFi-core
                    \begin{itemize}
                        \item Vedere se è possibile rifattorizzare la VM, facendo uso dei fields reificati
                        \item Spostare i fields reificati in una libreria a parte
                    \end{itemize}
          \end{itemize}
    \item Review:
          \begin{itemize}
              \color{teal}
              \item In RuFi-core
                    \begin{itemize}
                        \item 3.2 - Testati vari metodi per l'integrazione di RuFi in scala, optato per ScalaNative (6.5 ore)
                        \item 3.2 - Costruito un esempio funzionante che utilizza Scala Native e integra qualche funzione di Rust utilizzando l'iteroperabilità di C (5 ore)
                    \end{itemize}
                    \color{cyan}
              \item In ScaFi-core
                    \begin{itemize}
                        \item 2.3.4 - Implementata libreria apposita per i fields reificati (4 ore)
                    \end{itemize}
          \end{itemize}
\end{itemize}

\par\noindent\rule{\textwidth}{0.5pt}

\subsection*{Sprint 5}

\begin{itemize}
    \item Goals:
          \begin{itemize}
              \color{teal}
              \item In RuFi-core
                    \begin{itemize}
                        \item Implementare i costrutti del linguaggio
                    \end{itemize}
                    \color{cyan}
              \item In ScaFi-core
                    \begin{itemize}
                        \item Implementare i restanti costrutti e builtins
                    \end{itemize}
                    \color{teal}
                    \color{magenta}
              \item In RuFi-core-wrapper
                    \begin{itemize}
                        \item Definire un progetto Gradle+SBT+Cargo che integri RuFi in Scala
                        \item Partire con l'implementazione
                    \end{itemize}
                    \color{blue}

              \item In ScaFi-fields
                    \begin{itemize}
                        \item Aggiungere la dipendenza di ScaFi-core
                    \end{itemize}
          \end{itemize}
    \item Review:
          \begin{itemize}
              \color{teal}
              \item In RuFi-core
                    \begin{itemize}
                        \item 1.2.2 - Il linguaggio non è stato implementato per via di alcune limitazioni di Rust (4 ore)
                        \item 1.1.1 Esplorate alcune soluzioni per aggirare i problemi di Rust (8 ore)
                    \end{itemize}
                    \color{cyan}
              \item In ScaFi-core
                    \begin{itemize}
                        \item 2.1.2.3 - Implementato il costrutto \textit{branch} (3 ore)
                        \item 2.1.2.4 - Implementato il costrutto \textit{foldhood} (4 ore)
                        \item 2.1.2.5 - Implementati i builtins (3 ore)
                    \end{itemize}
                    \color{magenta}
              \item In RuFi-core-wrapper
                    \begin{itemize}
                        \item 3.3 - Definito un progetto SBT+Cargo (7 ore)
                        \item 3.2 - L'implementazione non è stata avviata per alcuni problemi di incompatibilità (5 ore)
                    \end{itemize}
                    \color{blue}

              \item In ScaFi-fields
                    \begin{itemize}
                        \item 2.4 - Riparata la semantic release (2 ore)
                    \end{itemize}
          \end{itemize}
\end{itemize}

\par\noindent\rule{\textwidth}{0.5pt}


\subsection*{Sprint 6}

\begin{itemize}
    \item Goals:
          \begin{itemize}
              \color{teal}
              \item In RuFi-core
                    \begin{itemize}
                        \item Esplorare soluzioni per l'implementazione dei costrutti del linguaggio
                    \end{itemize}
                    \color{magenta}
              \item In RuFi-core-wrapper
                    \begin{itemize}
                        \item Iniziare l'implementazione
                    \end{itemize}
                    \color{blue}

              \item In ScaFi-fields
                    \begin{itemize}
                        \item Aggiungere la dipendenza da ScaFi-core
                    \end{itemize}
          \end{itemize}
    \item Review:
          \begin{itemize}
              \color{blue}

              \item In ScaFi-fields
                    \begin{itemize}
                        \item 2.3.2 - Aggiunta la dipendenza a ScaFi-core (1 ora)
                        \item 2.3.2. - Implementati i costrutti fondamentali del linguaggio (8 ore)
                    \end{itemize}
              \item 3.2 - Sperimenti sull'interperabilità e costrutti del linguaggio in Rust hanno rivelato problemi complessi (7 ore)
          \end{itemize}
\end{itemize}

\par\noindent\rule{\textwidth}{0.5pt}


\subsection*{Sprint 7}

\begin{itemize}
    \item Goals:
          \begin{itemize}
              \color{teal}
              \item In RuFi-core
                    \begin{itemize}
                        \item Cambiare l'API della VM per renderla immutabile
                        \item Implementare i costrutti fondamentali del linguaggio
                    \end{itemize}
                    \color{cyan}
              \item In ScaFi-core
                    \begin{itemize}
                        \item Aggiungere test "TestByRound" e "TestByEquivalence"
                    \end{itemize}
                    \color{blue}

              \item In ScaFi-fields
                    \begin{itemize}
                        \item Controllare la correttezza di "TestByRound"
                        \item Aggiungere "TestByEquivalence"
                    \end{itemize}
          \end{itemize}
    \item Review:
          \begin{itemize}
              \color{teal}
              \item In RuFi-core
                    \begin{itemize}
                        \item 1.1.6 - Decisione di \textbf{non} rendere la RoundVM immutabile (3.5 ore)
                        \item 1.2.2 Implementati i costrutti fondamentali del linguaggio (8 ore)
                    \end{itemize}
                    \color{cyan}
              \item In ScaFi-core
                    \begin{itemize}
                        \item 2.2 - Aggiunti i test (5.5 ore)
                    \end{itemize}
                    \color{cyan}
              \item In ScaFi-core-fields
                    \begin{itemize}
                        \item 2.2 - "TestByRound" funziona correttamente ma deve essere espanso (5 ore)
                        \item 2.2 - "TestByEquivalence" non è stato implementato
                    \end{itemize}
          \end{itemize}
\end{itemize}

\par\noindent\rule{\textwidth}{0.5pt}


\subsection*{Sprint 8}

\begin{itemize}
    \item Goals:
          \begin{itemize}
              \color{teal}
              \item In RuFi-core
                    \begin{itemize}
                        \item Implementare test "TestByRound" e "TestByEquivalence"
                        \item Rifattorizzare \texttt{nest} in \texttt{nestIn} e \texttt{nestOut}
                    \end{itemize}
                    \color{blue}

              \item In ScaFi-fields
                    \begin{itemize}
                        \item Espandere "TestByRound"
                        \item Implementare "TestByEquivalence"
                    \end{itemize}
          \end{itemize}
    \item Review:
          \begin{itemize}
              \color{teal}
              \item In RuFi-core
                    \begin{itemize}
                        \item 1.3.1 - Implementato "TestByRound" (4 ore)
                        \item 1.2.2 - Rifattorizzata \texttt{nest} in \texttt{nestIn} e \texttt{nestOut} (3.5 ore)
                    \end{itemize}
                    \color{blue}

              \item In ScaFi-fields
                    \begin{itemize}
                        \item 2.2 - Espansa "TestByRound" (3 ore)
                        \item 2.2 - Implementata "TestByEquivalence" (4.5 ore)
                    \end{itemize}
          \end{itemize}
\end{itemize}

\par\noindent\rule{\textwidth}{0.5pt}


\subsection*{Sprint 9}

\begin{itemize}
    \item Goals:
          \begin{itemize}
              \color{teal}
              \item In RuFi-core
                    \begin{itemize}
                        \item Implementare "TestByEquivalence"
                    \end{itemize}
          \end{itemize}
    \item Review:
          \begin{itemize}
              \color{teal}
              \item In RuFi-core
                    \begin{itemize}
                        \item 1.3.2 - Implementato "TestByEquivalence" (4 ore)
                    \end{itemize}
          \end{itemize}
\end{itemize}


\end{document}

