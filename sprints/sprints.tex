
\documentclass[12pt, a4paper]{article}
\usepackage[utf8]{inputenc}
\usepackage{fontenc}
\usepackage{xcolor}
\usepackage{hyperref}
\usepackage[english]{babel}
\usepackage[inline]{enumitem}
\usepackage{graphicx}
\usepackage{cleveref}

\graphicspath{ {res/} }

\newcommand{\versionmajor}{0}
\newcommand{\versionminor}{1}
\newcommand{\versionpatch}{0}
\newcommand{\version}{\versionmajor.\versionminor.\versionpatch}

\title{\LARGE
    Rustfields \\ 
    \small
    Sprints
    }

\author{
    Angela Cortecchia \\ 
    \small 
    angela.cortecchia@studio.unibo.it
    \and
    Paolo Penazzi \\ 
    \small
    paolo.penazzi@studio.unibo.it
}

\date{\small }

\begin{document}
\maketitle
\newpage

\section*{Sprints}

\par\noindent\rule{\textwidth}{0.5pt}

\subsection*{Sprint 1}

\begin{itemize}
    \item Goals:
          \begin{itemize}
              \item In RuFi-core
                    \begin{itemize}
                        \item Implementare il concetto di VM Status
                    \end{itemize}
              \item In ScaFi-core
                    \begin{itemize}
                        \item Definire un API di alto livello
                    \end{itemize}
          \end{itemize}
    \item Review:
          \begin{itemize}
              \item In RuFi-core
                    \begin{itemize}
                        \item Implementato l'Abstract Syntax tree
                        \item Implementato il concetto di Export
                        \item Configurato il workflow di release semantica
                    \end{itemize}
              \item In ScaFi-core
                    \begin{itemize}
                        \item Implementati i concetti di Path e Slot
                    \end{itemize}
              \item Iniziato il lavoro inerente ai fields reificati
          \end{itemize}
\end{itemize}

\par\noindent\rule{\textwidth}{0.5pt}

\subsection*{Sprint 2}

\begin{itemize}
    \item Goals:
          \begin{itemize}
              \item In RuFi-core
                    \begin{itemize}
                        \item Implementare il Context
                        \item Iniziare l'implementazione del concetto di RoundVM
                    \end{itemize}
              \item In ScaFi-core
                    \begin{itemize}
                        \item Implementare il concetto di Export
                        \item Reificare i fields
                        \item Implementare RoundVM
                        \item Implementare un API per i fields reificati
                        \item Configurare release semantica
                    \end{itemize}
          \end{itemize}
    \item Review:
          \begin{itemize}
              \item In RuFi-core
                    \begin{itemize}
                        \item Implementato il Context
                        \item Iniziata l'implementazione della RoundVM
                        \item Riparato il comportamento di Path
                        \item Refattorizzata l'API di Context
                    \end{itemize}
              \item In ScaFi-core
                    \begin{itemize}
                        \item Subentrati problemi nella configurazione di release semantica
                        \item Implementati i fields reificati
                        \item Implementato Export
                        \item Implementato Context
                    \end{itemize}
          \end{itemize}
\end{itemize}

\par\noindent\rule{\textwidth}{0.5pt}

\subsection*{Sprint 3}

\begin{itemize}
    \item Goals:
          \begin{itemize}
              \item In RuFi-core
                    \begin{itemize}
                        \item Completata la RoundVM
                    \end{itemize}
              \item In ScaFi-core
                    \begin{itemize}
                        \item Iniziare ad implementare i costrutti del field calculus
                        \item Finire di implementare il concetto di RoundVM
                    \end{itemize}
          \end{itemize}
    \item Review:
          \begin{itemize}
              \item In RuFi-core
                    \begin{itemize}
                        \item Implementata la RoundVM
                    \end{itemize}
              \item In ScaFi-core
                    \begin{itemize}
                        \item Implementati i costrutti del field calculus \texttt{nbr} e \texttt{rep}
                        \item Deciso di fermare momentaneamente lo sviluppo di RoundVM per provare ad utilizzare quella sviluppata in RuFi-core
                    \end{itemize}
          \end{itemize}
\end{itemize}

\par\noindent\rule{\textwidth}{0.5pt}

\subsection*{Sprint 4}

\begin{itemize}
    \item Goals:
          \begin{itemize}
              \item In RuFi-core
                    \begin{itemize}
                        \item Impostare il progetto Java-Rust
                        \item Sperimentare diverse soluzioni per la comunicazione tra Java e Rust (JNI, Wasmer, ScalaNative)
                    \end{itemize}
              \item In ScaFi-core
                    \begin{itemize}
                        \item Vedere se è possibile rifattorizzare la VM, facendo uso dei fields reificati
                        \item Spostare i fields reificati in una libreria a parte
                    \end{itemize}
          \end{itemize}
    \item Review:
          \begin{itemize}
              \item In RuFi-core
                    \begin{itemize}
                        \item Testati vari metodi per l'integrazione di RuFi in scala, optato per ScalaNative
                        \item Costruito un esempio funzionante che utilizza Scala Native e integra qualche funzione di Rust utilizzando l'iteroperabilità di C
                    \end{itemize}
              \item In ScaFi-core
                    \begin{itemize}
                        \item Implementata libreria apposita per i fields reificati
                    \end{itemize}
          \end{itemize}
\end{itemize}

\par\noindent\rule{\textwidth}{0.5pt}

\subsection*{Sprint 5}

\begin{itemize}
    \item Goals:
          \begin{itemize}
              \item In RuFi-core
                    \begin{itemize}
                        \item Implementare i costrutti del linguaggio
                    \end{itemize}
              \item In ScaFi-core
                    \begin{itemize}
                        \item Implementare i restanti costrutti e builtins
                    \end{itemize}
              \item In RuFi-core-wrapper
                    \begin{itemize}
                        \item Definire un progetto Gradle+SBT+Cargo che integri RuFi in Scala
                        \item Partire con l'implementazione
                    \end{itemize}
              \item In ScaFi-fields
                    \begin{itemize}
                        \item Aggiungere la dipendenza di ScaFi-core
                    \end{itemize}
          \end{itemize}
    \item Review:
          \begin{itemize}
              \item In RuFi-core
                    \begin{itemize}
                        \item Il linguaggio non è stato implementato per via di alcune limitazioni di Rust
                        \item Esplorate alcune soluzioni per aggirare i problemi
                    \end{itemize}
              \item In ScaFi-core
                    \begin{itemize}
                        \item Implementati i rimanenti costrutti e builtins
                    \end{itemize}
              \item In RuFi-core-wrapper
                    \begin{itemize}
                        \item Definito un progetto SBT+Cargo
                        \item L'implementazione non è stata avviata per alcuni problemi di incompatibilità
                    \end{itemize}
              \item In ScaFi-fields
                    \begin{itemize}
                        \item Riparata la semantic release
                    \end{itemize}
          \end{itemize}
\end{itemize}

\par\noindent\rule{\textwidth}{0.5pt}


\subsection*{Sprint 6}

\begin{itemize}
    \item Goals:
          \begin{itemize}
              \item In RuFi-core
                    \begin{itemize}
                        \item Esplorare soluzioni per l'implementazione dei costrutti del linguaggio
                    \end{itemize}
              \item In RuFi-core-wrapper
                    \begin{itemize}
                        \item Iniziare l'implementazione
                    \end{itemize}
              \item In ScaFi-fields
                    \begin{itemize}
                        \item Aggiungere la dipendenza da ScaFi-core
                    \end{itemize}
          \end{itemize}
    \item Review:
          \begin{itemize}
              \item In ScaFi-fields
                    \begin{itemize}
                        \item Aggiunta la dipendenza a ScaFi-core
                        \item Implementati i costrutti fondamentali del linguaggio
                    \end{itemize}
              \item Sperimenti sull'interperabilità e costrutti del linguaggio in Rust hanno rivelato problemi complessi
          \end{itemize}
\end{itemize}

\par\noindent\rule{\textwidth}{0.5pt}


\subsection*{Sprint 7}

\begin{itemize}
    \item Goals:
          \begin{itemize}
              \item In RuFi-core
                    \begin{itemize}
                        \item Cambiare l'API della VM per renderla immutabile
                        \item Implementare i costrutti fondamentali del linguaggio
                    \end{itemize}
              \item In ScaFi-core
                    \begin{itemize}
                        \item Aggiungere test "TestByRound" e "TestByEquivalence"
                    \end{itemize}
              \item In ScaFi-fields
                    \begin{itemize}
                        \item Controllare la correttezza di "TestByRound"
                        \item Aggiungere "TestByEquivalence"
                    \end{itemize}
          \end{itemize}
    \item Review:
          \begin{itemize}
              \item In RuFi-core
                    \begin{itemize}
                        \item Decisione di \textbf{non} rendere la RoundVM immutabile
                        \item Implementati i costrutti fondamentali del linguaggio
                    \end{itemize}
              \item In ScaFi-core
                    \begin{itemize}
                        \item Aggiunti i test
                    \end{itemize}
              \item In ScaFi-core-fields
                    \begin{itemize}
                        \item "TestByRound" funziona correttamente ma deve essere espanso
                        \item "TestByEquivalence" non è stato implementato
                    \end{itemize}
          \end{itemize}
\end{itemize}

\par\noindent\rule{\textwidth}{0.5pt}


\subsection*{Sprint 8}

\begin{itemize}
    \item Goals:
          \begin{itemize}
              \item In RuFi-core
                    \begin{itemize}
                        \item Implementare i test "TestByRound" e "TestByEquivalence"
                        \item Rifattorizzare \texttt{nest} in \texttt{nestIn} e \texttt{nestOut}
                    \end{itemize}
              \item In ScaFi-fields
                    \begin{itemize}
                        \item Espandere "TestByRound"
                        \item Implementare "TestByEquivalence"
                    \end{itemize}
          \end{itemize}
    \item Review:
          \begin{itemize}
              \item In RuFi-core
                    \begin{itemize}
                        \item Rifattorizzata \texttt{nest} in \texttt{nestIn} e \texttt{nestOut}
                    \end{itemize}
              \item In ScaFi-fields
                    \begin{itemize}
                        \item Espansa "TestByRound"
                        \item Implementata "TestByEquivalence"
                    \end{itemize}
          \end{itemize}
\end{itemize}

\par\noindent\rule{\textwidth}{0.5pt}


\subsection*{Sprint 9}

\begin{itemize}
    \item Goals:
          \begin{itemize}
              \item In RuFi-core
                    \begin{itemize}
                        \item Implementare "TestByEquivalence"
                    \end{itemize}
          \end{itemize}
    \item Review:
          \begin{itemize}
              \item In RuFi-core
                    \begin{itemize}
                        \item Implementato "TestByEquivalence"
                    \end{itemize}
          \end{itemize}
\end{itemize}


\end{document}

